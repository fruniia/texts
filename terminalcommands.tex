\documentclass[10pt, a4paper]{article}
\usepackage[cm, empty]{fullpage}
\title{My terminal command cheatsheet}
\author{Anna Nilsson-Niia}
\begin{document}
\maketitle
\section*{Add new console, classlibrary, xunit, solution}
\begin{verbatim}
    dotnet new console
    dotnet new classlib
    dotnet new xunit
    dotnet new sln
\end{verbatim}
\section*{Add i.e. solution with name}
\begin{verbatim}
    dotnet new sln --name mySolution
\end{verbatim}
\section*{List all files in solution}
\begin{verbatim}
    dotnet sln list
\end{verbatim}
\section*{Adds one or more project to the solution file}
\begin{verbatim}
    dotnet sln mySolution.sln add projectpath/myProj
\end{verbatim}
\section*{Read file, listen, create file}
\begin{verbatim}
    cat 
    echo 
    touch
\end{verbatim}
\section*{Create new .git repository}
\begin{verbatim}
    git init
\end{verbatim}
\section*{Create a new repository on the command line}
\begin{verbatim}
    echo "# mygitrepository" >> README.md
    git init
    git add README.md
    git commit -m "first commit"
    git branch -M main
    git remote add origin git@github.com:fruniia/mygitrepository.git
    git push -u origin main
\end{verbatim}
\section*{Push an existing repository from the command line}
\begin{verbatim}
    git remote add origin git@github.com:fruniia/mygitrepository.git
    git branch -M main
    git push -u origin main
\end{verbatim}
\newpage
\section*{Commands in git}
\begin{verbatim}
    git status          show status of working directory
        -s          short version
    git add             adds file(s) to the staging area
    git commit          adds a commit to git
        -m          message
    git log             shows previous commits in branch
        --all
        --decorate
        --oneline
        --graph
    git show            take a closer look at commit
    git diff            shows changes between your working directory and index/staging area
\end{verbatim}
\section*{Help with a command}
\begin{verbatim}
    --h prefix works with most commands and will print the help text for said command
        e.g. diff --h   will print instructions for diff command 
                        (not to be confused with git diff)
    Notable exception:
        echo --h    will not print help on the echo command, instead try
        /bin/echo --help    
\end{verbatim}
\section*{Branching in git}
\begin{verbatim}
    branch              show and create branches
     -d <branch name>   delete branch          
    checkout            move HEAD to another branch or commit
     -b <branch name>   create new branch and change to that branch
                        does not change working directory
    merge               bring changes from one branch into another
\end{verbatim}
\section*{Remote in git}
\begin{verbatim}
    clone               clone a repo to a local .git repo     
    remote              show, create and change linked repos
    push                send local changes to remote
    pull                get changes from remote
    fetch               update local info about remote
\end{verbatim}
\section*{Undo in git}
When working in a shared repo it is important that the commit history doesn't change, instead
of removing the previous commit add the undos into a new commit.
\begin{verbatim}
    checkout            get a commit, undo all changes in workspace
    revert              create a new commit which undos all changes
    reset               remove files from staging or remove a LOCAL commit
        --soft
        --mixed
        --hard
    commit --amend      replaces the last commit with your new, improved commit
\end{verbatim}
\end{document}